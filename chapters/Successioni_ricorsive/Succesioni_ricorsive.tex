Quando abbiamo delle successioni definite per ricorrenza, dobbiamo trovare un modo per calcolare i termini successivi. Per fare ciò, possiamo usare il metodo di induzione.\\
Esempio 1\\
Consideriamo la successione definita per ricorrenza:
\[ 
\left\{
\begin{array}{ll}
a_{n} = 1 \\
a_{n+1} = a_{n} + 2
\end{array}
\right.
\]
Come prima cosa possiamo dimostrare se la successione puo convergere o meno ad un valore finito o infinito.\\
Chiamiamo questo valore :$\lambda$.\\
Quindi $n \to +\infty$, $a_{n} \to \lambda$.\\
Allora, se la successione converge, allora $\lambda$ deve essere uguale a $\lambda + 2$.\\
Ma, $\lambda = \lambda + 2$ non ha soluzioni reali.\\
Possiamo dire che la successione non \emph{converge}.\\
ma possiamo trovare il limite della successione.\\
Per trovare il limite della successione, possiamo usare il metodo di induzione.\\
Ora, dobbiamo trovare il termine generale della successione.\\ 
\textbf{Caso base:} $n = 1$\\
$a_{1} = 1$
\\ $a_{n{+}1} = a_{1{+}1}=\quad a_{1} + 2 = \quad 1 + 2 = 3$\\
Quindi $a_{2} > a_{1}$. \\
Pensiamo sia strettamente crescente ora possiamo andare avanti con la nostra induzione.\\
