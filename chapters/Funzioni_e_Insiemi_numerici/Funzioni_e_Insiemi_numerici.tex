    Presi due insiemi $X$ e $Y$ e $f$ una funzione che associa ad ogni elemento 
    di $X$ \emph{uno e uno solo} elemento di $Y$, la terna ordinata $(X,Y,f)$ è detta funzione.
    Il primo insieme è detto \emph{dominio} della funzione, il secondo \emph{codominio} e 
    l'insieme dei valori che la funzione assume è detto \emph{immagine} della funzione. Per indicare la funzione si usa la notazione:
        \begin{equation}
            f: X \to Y
        \end{equation}
        dove $x$ indica il generico elemento di $X$ e $f(x)$ il corrispondente elemento di $Y$.
        Se $f(x_0)=y_0$ si dice che $y_0$ è l'immagine di $x_0$ e si scrive $y_0=f(x_0)$. 
        \\\\ \emph{Definizione di prolungamento}: di una funzione: siano $X,Y,Z$ tre insiemi e $f: X \to Y$ e $g: Y \to Z$ due funzioni. Si dice che $g$ è il prolungamento di $f$ se:
        \begin{equation}
            g(x)=g(f(x)) \quad \forall x \in X
        \end{equation}
        \emph{Definizione di funzione inversa}: sia $f: X \to Y$ una funzione. Si dice che $f$ è invertibile se esiste una funzione $g: Y \to X$ tale che:
        \begin{equation}
            g(f(x))=x \quad \forall x \in X
        \end{equation}
    \emph{Definizione di funzione suriettiva}: si dice suriettiva se ogni elemento del codominio $Y$ ha almeno un elemento nel dominio $X$ che lo mappa su di esso:
        \begin{equation}
            \forall y \in Y \quad \exists x \in X \quad \text{tale che} \quad f(x)=y
        \end{equation}  
    \emph{Definizione di funzione iniettiva}: si dice iniettiva se ad ogni elemento del dominio $X$ corrisponde un solo elemento del codominio $Y$:
        \begin{equation}
            \forall x_1,x_2 \in X \quad \text{tali che} \quad x_1 \neq x_2 \quad \text{allora} \quad f(x_1) \neq f(x_2)
        \end{equation}
\emph{Diciamo che $f$ ha una corrispondenza biunivoca tra $X$ e $Y$ se è sia suriettiva che iniettiva.}

\subsection{Esempi di funzioni}
    \begin{itemize}
        \item $f(x)=x^2$ è una funzione suriettiva ma non iniettiva.
        \item $f(x)=x^3$ è una funzione suriettiva e iniettiva.
        \item $f(x)=\sin(x)$ è una funzione suriettiva ma non iniettiva.
        \item $f(x)=\cos(x)$ è una funzione suriettiva ma non iniettiva.
        \item $f(x)=\tan(x)$ è una funzione suriettiva ma non iniettiva.
        \item $f(x)=\log(x)$ è una funzione suriettiva ma non iniettiva.
        \item $f(x)=\exp(x)$ è una funzione suriettiva e iniettiva.
    \end{itemize}
\section{Relazioni di equivalenza}
    \label{sec:Relazioni_di_equivalenza} 
    Una relazione di equivalenza è una relazione binaria $\sim$ su un insieme $X$ che soddisfa le seguenti proprietà:
    \begin{itemize}
        \item \emph{Riflessiva}: $\forall x \in X, \; x \, \sim \, x$
        \item \emph{Simmetrica}: $\forall x, y \in X, \; x \, \sim \, y \implies y \, \sim \, x$
        \item \emph{Transitiva}: $\forall x, y, z \in X, \; x \, \sim \, y \land y \, \sim \, z \implies x \, \sim \, z$
    \end{itemize}
    \emph{Definizione di classe di equivalenza}: sia $\sim$ una relazione di equivalenza su un insieme $X$ e $x \in X$. La classe di equivalenza di $x$ rispetto a $\sim$ è l'insieme:
    \begin{equation}
        [x] = \{y \in X \; | \; y \, \sim \, x\}   
    \end{equation}
    \emph{Definizione di insieme quoziente}: sia $\sim$ una relazione di equivalenza su un insieme $X$. L'insieme quoziente di $X$ rispetto a $\sim$ è l'insieme:
    \begin{equation}
        X/\sim = \{[x] \; | \; x \in X\}
    \end{equation}
    \emph{Definizione di funzione di proiezione}: sia $\sim$ una relazione di equivalenza su un insieme $X$. La funzione di proiezione è la funzione:
    \begin{equation}
        \pi: X \to X/\sim \quad \text{tale che} \quad \pi(x) = [x]
    \end{equation}
    