\section{Serie Numeriche}
    Questo capitolo tratta delle serie numeriche, ovvero di somme infinite di numeri reali.
    
    
    \begin{definizione}
        
    Sia $\{a_n\}$ una successione di numeri reali. La somma parziale di ordine $n$ è definita come:
    \[
    S_n = \sum_{n=1}^n a_n
    \] si dice serie una successione di somme parziali, ovvero:
    \[ S = \sum_{n=1}^{+\infty} a_n \]
    Se esiste il limite della successione delle somme parziali, ovvero: \[ \lim_{n \to +\infty} S_n = S \] allora si dice che la serie converge e si scrive: \[ S = \sum_{k=1}^{+\infty} a_n \] altrimenti si dice che la serie diverge.
    Se la serie converge, si dice che la somma della serie è $S$, 
    Allora la successione $\{a_n\}$ tende a zero,
    Quindi la successione delle somme parziali è limitata.
    Quand'è che una serie converge?
    \end{definizione}
    \subsection{Criteri di Convergenza}
    \begin{itemize}
        \item Se la serie converge, allora la successione $\{a_n\}$ tende a zero.
        \item Se la serie converge, allora la successione delle somme parziali è limitata.
    \end{itemize}
    Per verificare queste condizioni la prima cosa da notare anche ad occhio è se la successione $\{a_n\}$ tende a zero. Se non tende a zero, la serie \emph{non può convergere} perchè sarebbe come sommare infiniti numeri diversi da zero.
    Quando $\{a_n\}$ tende a zero, la serie può comunque non convergere, quindi è condizione necessaria ma non sufficente.

\subsection{La serie geomentrica}
\begin{definizione}
    Una serie geometrica è una serie della forma:
    \begin{equation}
     \sum_{k=1}^{+\infty} x^k = 1 + x + x^2 + \ldots + x^{k} + \ldots 
    \end{equation}
\end{definizione}


    La serie geometrica converge se e solo se $|x| < 1$ e in tal caso la somma della serie è:
    \[ \sum_{n=1}^{+\infty} x^k = \frac{1-x^{n+1}}{1-x} \]

    perche facendo il limite si ha:
    \begin{equation}
                \lim_{n \to +\infty} \frac {1-x^{n+1}} {1-x} = \begin{cases}
                    \frac{1}{1-x} & \text{se } -1 < x < 1 \\
                    \text{non esiste} & \text{se } x \leq -1
                \end{cases}
    \end{equation}
    \begin{esempio}[Serie Geometrica]
        Studiamo la serie:
            \[ \sum_{n=1}^{+\infty} \left( \frac{1}{2} \right)^n \]
        Questa è una serie geometrica con $x = 1/2$, quindi converge perchè $|1/2| < 1$ e la somma della serie è:
        \[ \sum_{n=1}^{+\infty} \left( \frac{1}{2} \right)^n = \frac{1}{1-1/2} = 2 \]
    \end{esempio}
    \subsection{Criterio del Confronto}





        \subsection{Criterio del Rapporto}
             \begin{definizione}
                    Sia \emph{$\sum_{n=1}^{+\infty}a_n$} una serie a termini positivi e supponiamo esista il limite:
                    \[ \lim_{n \to +\infty} \frac{a_{n+1}}{a_n} = l \]
                    Allora: se $l < 1$ la serie converge, se $l > 1$ la serie diverge, se $l = 1$ il criterio non ci da alcuna indicazione sul carattere della serie.
            \end{definizione}
                    \emph{Spesso se questo criterio non ci da indicazione sul carattere della serie allora neanche il criterio della radice}.
        \begin{esempio}[Criterio del Rapporto]
            Studiamo la serie:
                \[ \sum_{n=1}^{+\infty} \frac{1}{n!} \]
            Applichiamo il criterio del rapporto:
            in questo caso si ha che:
            \begin{equation} \lim_{n \to +\infty} \frac{1/(n+1)!}{1/n!} = \lim_{n \to +\infty} \frac{1}{n+1} = 0
            \end{equation}
            Quindi per il criterio del rapporto $0<1$ quindi la serie converge.
        \end{esempio}
        \subsection{Criterio della Radice}
        \begin{definizione}
            Sia \emph{$\sum_{n=1}^{+\infty}a_n$} una serie a termini positivi e supponiamo esista il limite:
            \[ \lim_{n \to +\infty} \sqrt[n]{a_n} = l \]
            Allora: se $l < 1$ la serie converge, se $l > 1$ la serie diverge, se $l = 1$ il criterio non ci da alcuna indicazione sul carattere della serie.
        \end{definizione} 
        
    \emph{Spesso se questo criterio non ci da indicazione sul carattere della serie allora neanche il criterio del rapporto}


    \begin{esempio}[Criterio della Radice]
        Studiamo la serie:
            \[ \sum_{n=1}^{+\infty} \frac{1}{n^n} \]
        Applichiamo il criterio della radice:
        in questo caso si ha che:
        \begin{equation} \lim_{n \to +\infty} \sqrt[n]{1/n^n} = \lim_{n \to +\infty} \frac{1}{n} = 0
        \end{equation}
        Quindi per il criterio della radice $0<1$ quindi la serie converge.
            \end{esempio}
    \subsection{Criterio di Cauchy}
            \begin{definizione}
                    Sia $\{a_n\}$ una successione di numeri reali. La serie $\sum_{k=1}^{+\infty} a_k$ converge se e solo se per ogni $\varepsilon > 0$ esiste $n_0 \in \mathbb{N}$ tale che per ogni $n \geq m \geq n_0$ si ha:
                      \[ |a_{m+1} + a_{m+2} + \ldots + a_n| < \varepsilon \]
            \end{definizione}
    \begin{esempio}[Criterio di Cauchy]
        Studiamo la serie:
            \[ \sum_{n=1}^{+\infty} \frac{1}{n^2} \]
        Applichiamo il criterio di Cauchy:
        in questo caso si ha che:
        \begin{equation} |a_{m+1} + a_{m+2} + \ldots + a_n| = \left| \frac{1}{(m+1)^2} + \frac{1}{(m+2)^2} + \ldots + \frac{1}{n^2} \right| < \varepsilon
        \end{equation}
        Quindi per il criterio di Cauchy la serie converge.
    \end{esempio}