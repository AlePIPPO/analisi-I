\section{Serie Numeriche}
    Questo capitolo tratta delle serie numeriche, ovvero di somme infinite di numeri reali.
    
    
    \begin{definizione}
        
    Sia $\{a_n\}$ una successione di numeri reali. La somma parziale di ordine $n$ è definita come:
    \[
    S_n = \sum_{n=1}^n a_n
    \] si dice serie una successione di somme parziali, ovvero:
    \[ S = \sum_{n=1}^{+\infty} a_n \]
    Se esiste il limite della successione delle somme parziali, ovvero: \[ \lim_{n \to +\infty} S_n = S \] allora si dice che la serie converge e si scrive: \[ S = \sum_{k=1}^{+\infty} a_n \] altrimenti si dice che la serie diverge.
    Se la serie converge, si dice che la somma della serie è $S$, 
    Allora la successione $\{a_n\}$ tende a zero,
    Quindi la successione delle somme parziali è limitata.
    Quand'è che una serie converge?
    \end{definizione}
    \subsection{Condizioni Affinchè una Serie Converga}
    \begin{itemize}
        \item Se la serie converge, allora la successione $\{a_n\}$ tende a zero.
        \item Se la serie converge, allora la successione delle somme parziali è limitata.
    \end{itemize}
    Per verificare queste condizioni la prima cosa da notare anche ad occhio è se la successione $\{a_n\}$ tende a zero. Se non tende a zero, la serie \emph{non può convergere} perchè sarebbe come sommare infiniti numeri diversi da zero.
    Quando $\{a_n\}$ tende a zero, la serie può comunque non convergere, quindi è condizione necessaria ma non sufficente. 
        \begin{approfondimento}
            Se la serie di termine generale $a_k$ e $b_k$ sono regolari e se:
                \begin{equation}
                    \sum_{k=1}^{+\infty} a_k + \sum_{k=1}^{+\infty} b_k 
                \end{equation}
            ha significato nella retta estesa $\mathbb{R} \cup \{+\infty, -\infty\}$ allora:
                \begin{equation}
                    \sum_{k=1}^{+\infty} (a_k + b_k) = \sum_{k=1}^{+\infty} a_k + \sum_{k=1}^{+\infty} b_k 
                \end{equation}
        \end{approfondimento}
        \begin{approfondimento}
            se la serie di termine generale $a_k$ è regolare, anche la serie di termine generale $c \cdot a_k$ è regolare per ogni $c \in \mathbb{R}$ e si ha:
                \begin{equation}
                    \sum_{k=1}^{+\infty} c \cdot a_k = c \cdot \sum_{k=1}^{+\infty} a_k
                \end{equation}
        \end{approfondimento}
    \begin{definizione}
        Se le due serie convergono, allora la serie della somma dei termini converge e la somma della serie è la somma delle somme delle due serie.
        Se la serie converge, allora la serie moltiplicata per un numero reale converge e la somma della serie è il numero reale moltiplicato per la somma della serie.
    \end{definizione}
        

\subsection{La Serie Geomentrica}
        \begin{definizione}
            Una serie geometrica è una serie della forma:
             \begin{equation}
                \sum_{k=1}^{+\infty} x^k = 1 + x + x^2 + \ldots + x^{k} + \ldots 
            \end{equation}
        \end{definizione}


    La serie geometrica converge se e solo se $|x| < 1$ e in tal caso la somma della serie è:
    \[ \sum_{n=1}^{+\infty} x^k = \frac{1-x^{n+1}}{1-x} \]

    perche facendo il limite si ha:
    \begin{equation}
                \lim_{n \to +\infty} \frac {1-x^{n+1}} {1-x} = \begin{cases}
                    \frac{1}{1-x} & \text{se } -1 < x < 1 \\
                    \text{non esiste} & \text{se } x \leq -1
                \end{cases}
    \end{equation}
    \begin{esempio}[Serie Geometrica]
        Studiamo la serie:
            \[ \sum_{n=1}^{+\infty} \left( \frac{1}{2} \right)^n \]
        Questa è una serie geometrica con $x = 1/2$, quindi converge perchè $|1/2| < 1$ e la somma della serie è:
        \[ \sum_{n=1}^{+\infty} \left( \frac{1}{2} \right)^n = \frac{1}{1-\frac{1}{2}} = 2 \]
    \end{esempio}

\subsection{La Serie Armonica}
        \begin{definizione}
            La serie armonica è una serie della forma:
            \[ \sum_{n=1}^{+\infty} \frac{1}{n} = 1 + \frac{1}{2} + \frac{1}{3} + \ldots + \frac{1}{n} + \ldots \]
        \end{definizione}
        La serie armonica diverge, ovvero non converge, infatti:
        \[ \lim_{n \to +\infty} \sum_{k=1}^{n} \frac{1}{k} = +\infty \]
        \begin{esempio}[Serie Armonica]
            Studiamo la serie:
                \[ \sum_{n=1}^{+\infty} \frac{1}{n} \]
            Questa è una serie armonica, quindi diverge.
        \end{esempio} DIMOSTRAZIONE CON INTEGRALE DA METTERE DOPO AVERLI CAPITI
\newpage
\section{Criteri di Convergenza}
    \subsection{Criterio del Confronto}
        \begin{definizione}
            Siano $\{a_n\}$ e $\{b_n\}$ due successioni di numeri reali tali che $0 \leq a_n \leq b_n$ per ogni $n \in \mathbb{N}$. Allora:
            \begin{itemize}
                \item $\sum_{n=1}^{+\infty} b_n < +\infty$ $\to$ $\sum_{n=1}^{+\infty} a_n < + \infty$.
                \item $\sum_{n=1}^{+\infty} a_n = +\infty$ $\sum_{n=1}^{+\infty} b_n =+\infty$.
            \end{itemize}
        \end{definizione}
        Quindi se confrontiamo una serie $a_n$ con una serie $b_n$ e la serie $b_n$ converge, allora anche la serie $a_n$ converge. Se la serie $b_n$ diverge, allora anche la serie $a_n$ diverge.
        Per utilizzare il criterio del confronto è necessario trovare una serie $b_n$ che sia più semplice da trattare rispetto alla serie $a_n$.
        Per farlo possiamo usare i limiti quindi:
            \begin{equation}
                \lim_{n \to +\infty} \frac{a_n}{b_n} = l
            \end{equation}
        \begin{itemize}
            \item se $ 0 < l < +\infty$ allora $\sum_{n=1}^{+\infty} a_n$ e $\sum_{n=1}^{+\infty} b_n$ hanno lo stesso carattere(in genere si sceglie una serie $b_n$ convergente per "ingabbiare" la prima).
            \item se $l = 0$ allora $\sum_{n=1}^{+\infty} b_n$ converge $\to$ $\sum_{n=1}^{+\infty} a_n$ converge.
            \item se $l = +\infty$ allora $\sum_{n=1}^{+\infty} b_n$ diverge $\to$ $\sum_{n=1}^{+\infty} a_n$ diverge.
        \end{itemize}
                \begin{approfondimento}
                    se la serie $b_n$ diverge si possono fare due cose:
                        \begin{itemize}
                            \item se $a_n$ è "infinito di ordine superiore" rispetto alla serie $b_n$ allora se il limite tende a zero allora la serie $a_n$ diverge.
                            \item se $a_n$ è "infinito di ordine inferiore" rispetto alla serie $b_n$ allora se il limite tende a zero allora la serie $a_n$ converge.
                        \end{itemize}
                        quindi per $l=0$ si ha che la serie $a_n$ diverge e, per $l=+\infty$ si ha che la serie $a_n$ converge.
                \end{approfondimento}
            \begin{approfondimento}
                se $l=0$ diamo che la serie $a_n$ converge perchè la serie $b_n$ è "infinito di ordine superiore" rispetto alla serie $a_n$, quindi se il limite tende a zero, visto che la serie $b_n$ converge, allora anche la serie $a_n$ converge(perchè è "più piccola" della $b_n$). Con lo stesso regionamento possiamo spiegare perchè se il limite fa $l=+\infty$ allora la serie $a_n$ diverge.
            \end{approfondimento}
        \begin{esempio}[Criterio del Confronto]
            Studiamo la serie:
                \[ \sum_{n=1}^{+\infty} \frac{1}{n^2} \]
            Questa è una serie armonica, quindi diverge. Ma possiamo confrontarla con la serie armonica:
                \[ \sum_{n=1}^{+\infty} \frac{1}{n^2} \leq \sum_{n=1}^{+\infty} \frac{1}{n} \]
                    \begin{equation}
                        \lim_{n \to +\infty} \frac{\frac{1}{n^2}}{\frac{1}{n}} = \lim_{n \to +\infty} \frac{1}{n^2} \cdot n  = \lim_{n \to +\infty} \frac{1}{n}=0
                    \end{equation}
            Quindi la serie $\sum_{n=1}^{+\infty} \frac{1}{n^2}$ diverge.
        \end{esempio}
    \subsection{Criterio del Rapporto}
             \begin{definizione}
                    Sia \emph{$\sum_{n=1}^{+\infty}a_n$} una serie a termini positivi e supponiamo esista il limite:
                    \[ \lim_{n \to +\infty} \frac{a_{n+1}}{a_n} = l \]
                    Allora: se $l < 1$ la serie converge, se $l > 1$ la serie diverge, se $l = 1$ il criterio non ci da alcuna indicazione sul carattere della serie.
            \end{definizione}
                    \emph{Spesso se questo criterio non ci da indicazione sul carattere della serie allora neanche il criterio della radice}.
        \begin{esempio}[Criterio del Rapporto]
            Studiamo la serie:
                \[ \sum_{n=1}^{+\infty} \frac{1}{n!} \]
            Applichiamo il criterio del rapporto:
            in questo caso si ha che:
            \begin{equation} \lim_{n \to +\infty} \frac{1/(n+1)!}{1/n!} = \lim_{n \to +\infty} \frac{1}{n+1} = 0
            \end{equation}
            Quindi per il criterio del rapporto $0<1$ quindi la serie converge.
        \end{esempio}
    \subsection{Criterio della Radice}
        \begin{definizione}
            Sia \emph{$\sum_{n=1}^{+\infty}a_n$} una serie a termini positivi e supponiamo esista il limite:
            \[ \lim_{n \to +\infty} \sqrt[n]{a_n} = l \]
            Allora: se $l < 1$ la serie converge, se $l > 1$ la serie diverge, se $l = 1$ il criterio non ci da alcuna indicazione sul carattere della serie.
        \end{definizione} 
        
    \emph{Spesso se questo criterio non ci da indicazione sul carattere della serie allora neanche il criterio del rapporto}


        \begin{esempio}[Criterio della Radice]
        Studiamo la serie:
            \[ \sum_{n=1}^{+\infty} \frac{1}{n^n} \]
        Applichiamo il criterio della radice:
        in questo caso si ha che:
        \begin{equation} \lim_{n \to +\infty} \sqrt[n]{1/n^n} = \lim_{n \to +\infty} \frac{1}{n} = 0
        \end{equation}
        Quindi per il criterio della radice $0<1$ quindi la serie converge.
            \end{esempio}
    \subsection{Criterio di Cauchy}
            \begin{definizione}
                    Sia $\{a_n\}$ una successione di numeri reali. La serie $\sum_{k=1}^{+\infty} a_k$ converge se e solo se per ogni $\varepsilon > 0$ esiste $n_0 \in \mathbb{N}$ tale che per ogni $n \geq m \geq n_0$ si ha:
                      \[ |a_{m+1} + a_{m+2} + \ldots + a_n| < \varepsilon \]
            \end{definizione}
    \begin{esempio}[Criterio di Cauchy]
        Studiamo la serie:
            \[ \sum_{n=1}^{+\infty} \frac{1}{n^2} \]
        Applichiamo il criterio di Cauchy:
        in questo caso si ha che:
        \begin{equation} |a_{m+1} + a_{m+2} + \ldots + a_n| = \left| \frac{1}{(m+1)^2} + \frac{1}{(m+2)^2} + \ldots + \frac{1}{n^2} \right| < \varepsilon
        \end{equation}
        Quindi per il criterio di Cauchy la serie converge.
    \end{esempio}
\section{Serie Alternate}
    \begin{definizione}
        Una serie $\sum_{n=1}^{+\infty} (-1)^n a_n$ si dice serie alternata.
    \end{definizione}
        \begin{approfondimento}
            Una serie alternata è una serie in cui i termini sono alternativamente positivi e negativi.
                \begin{equation}
                    \sum_{n=1}^{+\infty} (-1)^n a_n = a_1 - a_2 + a_3 - a_4 + \ldots
                \end{equation}
        \end{approfondimento}
    \subsection{Criterio di Leibniz}
    Il criterio di Leibniz è un criterio di convergenza per le serie alternate, ovvero serie in cui i termini sono alternativamente positivi e negativi. Questo criterio ci dice che se i termini della serie sono decrescenti e il loro limite tende a zero allora la serie converge.
        \begin{definizione}[Criterio di Leibniz]
            Sia $\{a_n\}$ una successione di numeri reali tali che:
            \begin{itemize}
                \item $a_n \geq 0$ per ogni $n \in \mathbb{N}$.
                \item $\{a_n\}$ sia decrescente.
                \item $\lim_{n \to +\infty} a_n = 0$.
            \end{itemize}
            Allora la serie alternata $\sum_{n=1}^{+\infty} (-1)^n a_n$ converge.
        \end{definizione}
        \begin{esempio}[Criterio di Leibniz]
            Studiamo la serie:
                \[ \sum_{n=1}^{+\infty} \frac{(-1)^n}{n} \]
            Applichiamo il criterio di Leibniz:
            in questo caso si ha che:
            \begin{itemize}
                \item $a_n = \frac{1}{n}$ è decrescente.
                \item $\lim_{n \to +\infty} \frac{1}{n} = 0$.
            \end{itemize}
            Quindi per il criterio di Leibniz la serie converge.
                \begin{approfondimento}
                    in questo caso possiamo verificare $\frac{1}{n}$ è decrescente perchè oltre a verificarlo ad occhio possiamo calcolare la derivata della funzione associata e vedere che è negativa.
                        \begin{equation}
                            f(x) = \frac{1}{x} \quad f'(x) = -\frac{1}{x^2} < 0
                        \end{equation}
                \end{approfondimento}
        \end{esempio}
\section{Convergenza Assoluta}
    \begin{definizione}
        Una serie $\sum_{n=1}^{+\infty} a_n$ si dice assolutamente convergente se la serie $\sum_{n=1}^{+\infty} |a_n|$ converge.
    \end{definizione}
    \begin{approfondimento}
        Se una serie converge assolutamente, allora converge anche la serie stessa.
    \end{approfondimento}
    \begin{esempio}[Convergenza Assoluta]
        Studiamo la serie:
            \[ \sum_{n=1}^{+\infty} \frac{(-1)^n}{n} \]
        Questa è una serie alternata, quindi converge per il criterio di Leibniz. Ma se prendiamo il valore assoluto dei termini:
            \[ \sum_{n=1}^{+\infty} \left| \frac{(-1)^n}{n} \right| = \sum_{n=1}^{+\infty} \frac{1}{n} \]
        Questa è una serie armonica, quindi diverge. Quindi la serie $\sum_{n=1}^{+\infty} \frac{(-1)^n}{n}$ converge ma non converge assolutamente.
            \begin{approfondimento}
                togliamo $n$ a esponente perchè senza il segno meno dell'uno risulta comunque uno($1^n$).
            \end{approfondimento}
    \end{esempio}
    \begin{approfondimento}
        Se una serie converge assolutamente, allora converge anche la serie stessa. Ma se una serie converge, non è detto che converga anche assolutamente.
    \end{approfondimento}
\section{Esercizi Svolti Sulle Serie}
    \begin{esempio}
        Studiamo la serie:
        \[ \sum_{n=1}^{+\infty} \frac{2^n}{n^5} \]
        Applichiamo il criterio della radice:
        in questo caso si ha che:
        \begin{equation} \lim_{n \to +\infty} \sqrt[n]{\frac{2^n}{n^5}} = \lim_{n \to +\infty} \frac{2}{n^{5/n}} = \lim_{n \to +\infty} \frac{2}{n^{5/n}} = 0
        \end{equation}
    \end{esempio}