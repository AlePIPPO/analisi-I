Visto che \emph{$\mathbb{R}$ non è algebricamente chiuso}(vale a dire che non tutti i polinomi hanno radici in $\mathbb{R}$).\`E possibile estendere il campo dei numeri reali in modo da includere le radici di tutti i polinomi. Questo campo è il campo dei numeri complessi, indicato con $\mathbb{C}$.
\section{Definizione}
Per definire un campo, vanno a loro volta definite 2 oparazioni e soddisfatte 9 proprioetà.
\begin{itemize}
    \item $(a,b) + (c,d) = (a+c,b+d)$ $\to$ operazione di $\mathbb{R}$ già nota.
    \item $(a,b) * (c,d) = (ac-bd,ad+bc)$
    \item elemento neutro $0$ = (1,0)
    \item inverso? $\to$ prendiamo $(a,b) \neq (0,0) \to \exists (x,y) : (a,b)(x,y)=(1,0) \to (ax-by,ay+bx)=(1,0) \to ax-by=1 \land ay+bx=0 \to x=\frac{a}{a^2+b^2} \land y=\frac{-b}{a^2+b^2}$
\end{itemize}