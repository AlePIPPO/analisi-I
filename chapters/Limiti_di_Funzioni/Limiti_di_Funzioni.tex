Il concetto di limite è la base per lo studio delle qualita qualitative delle funzioni
matematiche. In questo capitolo si studierà il concetto di limite di una funzione in un
punto, che porterà il concetto di continuità di una funzione in un punto.
\subsection{intervalli limitati}
\label{subsec:intervalli_limitati}
abbiamo 4 tipi di intervalli limitati:
\begin{itemize}
    \item $[a,b] = \{x \in \mathbb{R} : a \leq x \leq b\}$
    \item $(a,b) = \{x \in \mathbb{R} : a < x < b\}$
    \item $[a,b) = \{x \in \mathbb{R} : a \leq x < b\}$
    \item $(a,b] = \{x \in \mathbb{R} : a < x \leq b\}$
\end{itemize}
\subsection{intervalli illimitati}
\label{subsec:intervalli_illimitati}
abbiamo 5 tipi di intervalli illimitati:
\begin{itemize}
    \item $[a,+\infty) = \{x \in \mathbb{R} : a \leq x\}$
    \item $(-\infty,b] = \{x \in \mathbb{R} : x \leq b\}$
    \item $(-\infty,+\infty) = \mathbb{R}$
    \item $[a,+\infty) = \{x \in \mathbb{R} : x \geq a\}$
    \item $(-\infty,b) = \{x \in \mathbb{R} : x < b\}$
    \end{itemize}

\subsection{Teorema del Confronto}
\label{subsec:teorema_del_confronto}

Esempio di teorema del Confronto:
\begin{equation}
    \lim_{x \to +\infty} \frac{\sin x}{x}    
    \end{equation}
    Visto che i valori del $sin(x)$ sono compresi tra -1 e 1 per $\forall x \in \mathbb{R}$, possiamo scrivere:
    \begin{equation}
        -1 \leq \sin x \leq 1 \Rightarrow -\frac{1}{x} \leq \frac{\sin x}{x} \leq \frac{1}{x}
        \end{equation}
        Per il teorema del confronto, se:
        \begin{equation}
            \lim_{x \to +\infty} -\frac{1}{x} = 0 \quad \lim_{x \to +\infty} \frac{1}{x} = 0
            \end{equation}
            allora:
            \begin{equation}
                \lim_{x \to +\infty} \frac{\sin x}{x} = 0
                \end{equation}