\section{Successioni Monotòne}
    \begin{definizione}
        Una successione $\{a_n\}$ si dice \emph{monotòna crescente} se per ogni $n \in \mathbb{N}$ si ha $a_n \leq a_{n+1}$, ovvero se i termini della successione sono in ordine crescente.
    \end{definizione}    
    \begin{definizione}
        Analogamente si dice \emph{monotòna decrescente} se per ogni $n \in \mathbb{N}$ si ha $a_n \geq a_{n+1}$, ovvero se i termini della successione sono in ordine decrescente.
    \end{definizione}
    Quando cerchiamo di capire se la nostra successione di termine generale $a_n$ è monotòna crescente o decrescente una delle possibilità è quella di studiare la funzione associata alla successione.
    \begin{definizione}
        Sia $\{a_n\}$ una successione di termini reali e sia $\psi: \mathbb{N} \to \mathbb{R}$ una funzione reale. Se $\psi(x) = a_x$ per ogni $n \in \mathbb{N}$, allora la successione $\{a_n\}$ è monotòna crescente se e solo se la funzione $\psi$ è crescente.
    \end{definizione}
    Si applica un ragionamento analogo per le successioni monotòne decrescenti.
    Per calcolare la monotònia di una (funzione)successione possiamo anche utilizzare il concetto di derivata.
    Quindi al fine di studiare la monotònia di una successione possiamo:
    \begin{itemize}
        \item Studiare la successione direttamente.
        \item Studiare la funzione associata alla successione.
        \item Studiare la derivata della funzione associata alla successione.
    \end{itemize}
    \begin{esempio}
        Studiamo la monotònia della successione $\{a_n\}$ definita da $a_n = n^2$.
        \begin{itemize}
            \item Studiamo la successione direttamente: $a_n = n^2$ è una successione crescente.
            \item Studiamo la funzione associata alla successione: $\psi(x) = x^2$ è una funzione crescente.
            \item Studiamo la derivata della funzione associata alla successione: $\psi'(x) = 2x$ è una funzione crescente.
        \end{itemize}
        studiando il segno della derivata delle $\psi'(x)$ possiamo concludere che la successione $\{a_n\}$ è crescente. Perchè la disequaizone $2x > 0$ è verificata per ogni $x > 0$.
            \begin{approfondimento}
                riconsiderando la funzione come una successione possiamo facilmete verificare anche la successione visto che i termini $x$ della funzione sono numeri naturali, Quindi viene verificata anche per tutte le $n$ della $a_n$ Saltando il primo termine $n=0$.
            \end{approfondimento}
    \end{esempio}
    \begin{esempio}
        Studiamo la monotònia della successione $\{a_n\}$ definita da $a_n = \frac{1}{n}$.
        \begin{itemize}
            \item Studiamo la successione direttamente: $a_n = \frac{1}{n}$ è una successione decrescente.
            \item Studiamo la funzione associata alla successione: $\psi(x) = \frac{1}{x}$ è una funzione decrescente.
            \item Studiamo la derivata della funzione associata alla successione: $\psi'(x) = -\frac{1}{x^2}$ è una funzione decrescente.
        \end{itemize}
    \end{esempio}