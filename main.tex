\documentclass[a4paper,12pt]{report} % Tipo di documento
\usepackage[utf8]{inputenc}           % Codifica
\usepackage[T1]{fontenc}              % Font moderni
\usepackage[italian]{babel}
\usepackage{amsmath,amssymb,amsfonts} % Pacchetti per matematica
%\usepackage{siunitx}                  % Unità di misura
\usepackage{physics}                  % Simboli fisici
\usepackage{graphicx}                 % Inserire immagini
\usepackage{pgfplots}
\pgfplotsset{compat=1.18}
\usepackage{ambienti}
\usepackage{bookmark}
\usepackage{float}                    % Gestione di immagini/flottanti
\usepackage{xcolor}                   % Colori personalizzati
\usepackage{hyperref}                 % Link cliccabili
\usepackage{geometry}                 % Personalizzazione margini
\geometry{margin=2.5cm}               % Margini di 2.5 cm
\usepackage{tikz}                     % Disegni e diagrammi
\usepackage[version=4]{mhchem}                   % Chimica, per formule come \ce{H2O}
\usepackage{booktabs}                 % Tabelle migliorate
\usepackage{cancel}                   % Per cancellare termini (\cancel{})
\usepackage{enumitem}                 % Liste personalizzate
%\usepackage{multicol}                 % Layout a colonne
\usepackage{fancyhdr}                 % Intestazioni e piè di pagina
\setlength{\headheight}{14.5pt}
\pagestyle{fancy}                     % Abilita intestazioni/piè di pagina
\fancyhead[L]{Appunti di Analisi I}      % Testo a sinistra nell'intestazione
\fancyhead[R]{\today}                 % Data a destra
\fancyfoot[C]{\thepage}               % Numerazione delle pagine al centro
\renewcommand{\baselinestretch}{1.2}  % Interlinea
\usepackage{xcolor}
\DeclareEmphSequence{\bfseries\itshape}

% Ensure all environments are properly closed
\usepackage{mdframed}
\newcommand{\mean}[1]{\langle #1 \rangle}
\setlength{\parindent}{0pt}           % Rientro automatico
\setlength{\parskip}{4pt}   
\title{Appunti di Analisi I}
% Ensure no unclosed mdframed environment
\begin{document}

\date{Anno 2024/2025}


    \begin{titlepage}
    \begin{center}
        \begin{minipage}[ht!]{0.49\textwidth}
                \includegraphics[width=\textwidth]{images/Logo+UniCT.png}
        \end{minipage}
        \hfill
        \begin{minipage}[th!]{0.49\textwidth}
                \includegraphics[width=\textwidth]{images/LogoDfa2.png}
        \end{minipage}
        \large
        \textbf{CORSO DI LAUREA IN FISICA}
        \vspace{0.5cm}
        \hrule
        \vspace{3cm}
        \centering
        \textbf{Alessandro Consoli\\Sonia Facciolà\\Joey Butchers}\\
        \vspace{1cm}
        \Huge{\textbf{Appunti di Analisi I}}\\
        
        \vfill
        
        \begin{minipage}[h]{0.35\textwidth}
            \hrule
            \vspace{0.3cm}
            \small \centering
            Appunti di Analisi I
            \vspace{0.35cm}
            \hrule
        \end{minipage}
        
        \vfill
        
        \hrule
        \vspace{0.3cm}
        \normalsize
        \textbf{Anno Accademico 2024/2025}
    \end{center}

    
\end{titlepage}
    
    \tableofcontents


    \maketitle
    
\chapter{Funzioni e Insiemi numerici}

    \section{Introduzione}

        Presi due insiemi $X$ e $Y$ e $f$ una funzione che associa ad ogni elemento 
    di $X$ \emph{uno e uno solo} elemento di $Y$, la terna ordinata $(X,Y,f)$ è detta funzione.
    Il primo insieme è detto \emph{dominio} della funzione, il secondo \emph{codominio} e 
    l'insieme dei valori che la funzione assume è detto \emph{immagine} della funzione. Per indicare la funzione si usa la notazione:
        \begin{equation}
            f: X \to Y
        \end{equation}
        dove $x$ indica il generico elemento di $X$ e $f(x)$ il corrispondente elemento di $Y$.
        Se $f(x_0)=y_0$ si dice che $y_0$ è l'immagine di $x_0$ e si scrive $y_0=f(x_0)$. 
        \\\\ \emph{Definizione di prolungamento}: di una funzione: siano $X,Y,Z$ tre insiemi e $f: X \to Y$ e $g: Y \to Z$ due funzioni. Si dice che $g$ è il prolungamento di $f$ se:
        \begin{equation}
            g(x)=g(f(x)) \quad \forall x \in X
        \end{equation}
        \emph{Definizione di funzione inversa}: sia $f: X \to Y$ una funzione. Si dice che $f$ è invertibile se esiste una funzione $g: Y \to X$ tale che:
        \begin{equation}
            g(f(x))=x \quad \forall x \in X
        \end{equation}
    \emph{Definizione di funzione suriettiva}: si dice suriettiva se ogni elemento del codominio $Y$ ha almeno un elemento nel dominio $X$ che lo mappa su di esso:
        \begin{equation}
            \forall y \in Y \quad \exists x \in X \quad \text{tale che} \quad f(x)=y
        \end{equation}  
    \emph{Definizione di funzione iniettiva}: si dice iniettiva se ad ogni elemento del dominio $X$ corrisponde un solo elemento del codominio $Y$:
        \begin{equation}
            \forall x_1,x_2 \in X \quad \text{tali che} \quad x_1 \neq x_2 \quad \text{allora} \quad f(x_1) \neq f(x_2)
        \end{equation}
\emph{Diciamo che $f$ ha una corrispondenza biunivoca tra $X$ e $Y$ se è sia suriettiva che iniettiva.}

\subsection{Esempi di funzioni}
    \begin{itemize}
        \item $f(x)=x^2$ è una funzione suriettiva ma non iniettiva.
        \item $f(x)=x^3$ è una funzione suriettiva e iniettiva.
        \item $f(x)=\sin(x)$ è una funzione suriettiva ma non iniettiva.
        \item $f(x)=\cos(x)$ è una funzione suriettiva ma non iniettiva.
        \item $f(x)=\tan(x)$ è una funzione suriettiva ma non iniettiva.
        \item $f(x)=\log(x)$ è una funzione suriettiva ma non iniettiva.
        \item $f(x)=\exp(x)$ è una funzione suriettiva e iniettiva.
    \end{itemize}
\section{Relazioni di equivalenza}
    \label{sec:Relazioni_di_equivalenza} 
    Una relazione di equivalenza è una relazione binaria $\sim$ su un insieme $X$ che soddisfa le seguenti proprietà:
    \begin{itemize}
        \item \emph{Riflessiva}: $\forall x \in X, \; x \, \sim \, x$
        \item \emph{Simmetrica}: $\forall x, y \in X, \; x \, \sim \, y \implies y \, \sim \, x$
        \item \emph{Transitiva}: $\forall x, y, z \in X, \; x \, \sim \, y \land y \, \sim \, z \implies x \, \sim \, z$
    \end{itemize}
    \emph{Definizione di classe di equivalenza}: sia $\sim$ una relazione di equivalenza su un insieme $X$ e $x \in X$. La classe di equivalenza di $x$ rispetto a $\sim$ è l'insieme:
    \begin{equation}
        [x] = \{y \in X \; | \; y \, \sim \, x\}   
    \end{equation}
    \emph{Definizione di insieme quoziente}: sia $\sim$ una relazione di equivalenza su un insieme $X$. L'insieme quoziente di $X$ rispetto a $\sim$ è l'insieme:
    \begin{equation}
        X/\sim = \{[x] \; | \; x \in X\}
    \end{equation}
    \emph{Definizione di funzione di proiezione}: sia $\sim$ una relazione di equivalenza su un insieme $X$. La funzione di proiezione è la funzione:
    \begin{equation}
        \pi: X \to X/\sim \quad \text{tale che} \quad \pi(x) = [x]
    \end{equation}
    

\chapter{Campo Complesso}

    \section{Numeri Complessi}

    Visto che \emph{$\mathbb{R}$ non è algebricamente chiuso}(vale a dire che non tutti i polinomi hanno radici in $\mathbb{R}$).\`E possibile estendere il campo dei numeri reali in modo da includere le radici di tutti i polinomi. Questo campo è il campo dei numeri complessi, indicato con $\mathbb{C}$.
\section{Definizione}
Per definire un campo, vanno a loro volta definite 2 oparazioni e soddisfatte 9 proprioetà.\\
\begin{itemize}
    \item $(a,b) + (c,d) = (a+c,b+d)$ $\to$ operazione di $\mathbb{R}$ già nota.
    \item $(a,b) * (c,d) = (ac-bd,ad+bc)$
    \item elemento neutro $0$ = (1,0)
    \item inverso? $\to$ prendiamo $(a,b) \neq (0,0) \to \exists (x,y) : (a,b)(x,y)=(1,0) \to (ax-by,ay+bx)=(1,0) \to ax-by=1 \land ay+bx=0 \to x=\frac{a}{a^2+b^2} \land y=\frac{-b}{a^2+b^2}$
\end{itemize}


\chapter{Limiti di Funzioni}

    \section{Limiti di Funzioni}

    \input{chapters/limiti_di_funzioni/limiti_di_funzioni.tex}



\chapter{Funzioni Continue}

    \section{Funzioni Continue}
    
    


Per definire una funzione continua e parlare quindi continuità di una funzione dobbiamo
definire delle condizioni che la funzione dovrà rispettare per essere definita continua in un punto: partendo da questo preambolo:
    \begin{equation}
        f(x):X \to Y \quad x_0 \in X
    \end{equation}
le condizioni sono:
    \begin{equation}
        \text{1)} \exists f(x_0) \qquad \text{2)} \exists \lim_{x \to x_0} f(x)=l \in \mathbb{R} \qquad \text{3)} f(x_0)=l
    \end{equation}
    
    
\chapter{Successioni}
    
    \section{Successioni Monotòne}
    \begin{definizione}
        Una successione $\{a_n\}$ si dice \emph{monotòna crescente} se per ogni $n \in \mathbb{N}$ si ha $a_n \leq a_{n+1}$, ovvero se i termini della successione sono in ordine crescente.
    \end{definizione}    
    \begin{definizione}
        Analogamente si dice \emph{monotòna decrescente} se per ogni $n \in \mathbb{N}$ si ha $a_n \geq a_{n+1}$, ovvero se i termini della successione sono in ordine decrescente.
    \end{definizione}
    Quando cerchiamo di capire se la nostra successione di termine generale $a_n$ è monotòna crescente o decrescente una delle possibilità è quella di studiare la funzione associata alla successione.
    \begin{definizione}
        Sia $\{a_n\}$ una successione di termini reali e sia $\psi: \mathbb{N} \to \mathbb{R}$ una funzione reale. Se $\psi(x) = a_x$ per ogni $n \in \mathbb{N}$, allora la successione $\{a_n\}$ è monotòna crescente se e solo se la funzione $\psi$ è crescente.
    \end{definizione}
    Si applica un ragionamento analogo per le successioni monotòne decrescenti.
    Per calcolare la monotònia di una (funzione)successione possiamo anche utilizzare il concetto di derivata.
    Quindi al fine di studiare la monotònia di una successione possiamo:
    \begin{itemize}
        \item Studiare la successione direttamente.
        \item Studiare la funzione associata alla successione.
        \item Studiare la derivata della funzione associata alla successione.
    \end{itemize}
    \begin{esempio}
        Studiamo la monotònia della successione $\{a_n\}$ definita da $a_n = n^2$.
        \begin{itemize}
            \item Studiamo la successione direttamente: $a_n = n^2$ è una successione crescente.
            \item Studiamo la funzione associata alla successione: $\psi(x) = x^2$ è una funzione crescente.
            \item Studiamo la derivata della funzione associata alla successione: $\psi'(x) = 2x$ è una funzione crescente.
        \end{itemize}
        studiando il segno della derivata delle $\psi'(x)$ possiamo concludere che la successione $\{a_n\}$ è crescente. Perchè la disequaizone $2x > 0$ è verificata per ogni $x > 0$.
            \begin{approfondimento}
                riconsiderando la funzione come una successione possiamo facilmete verificare anche la successione visto che i termini $x$ della funzione sono numeri naturali, Quindi viene verificata anche per tutte le $n$ della $a_n$ Saltando il primo termine $n=0$.
            \end{approfondimento}
    \end{esempio}
    \begin{esempio}
        Studiamo la monotònia della successione $\{a_n\}$ definita da $a_n = \frac{1}{n}$.
        \begin{itemize}
            \item Studiamo la successione direttamente: $a_n = \frac{1}{n}$ è una successione decrescente.
            \item Studiamo la funzione associata alla successione: $\psi(x) = \frac{1}{x}$ è una funzione decrescente.
            \item Studiamo la derivata della funzione associata alla successione: $\psi'(x) = -\frac{1}{x^2}$ è una funzione decrescente.
        \end{itemize}
    \end{esempio}

%\chapter{limiti notevoli, derivate notevoli, sviluppi di Taylor notevoli}

%limiti notevoli
\section{Limiti Notevoli}   
$\lim\limits_{x \to 0} \frac{\sin x}{x} = 1$ \qquad \qquad \qquad \qquad
$\lim\limits_{x \to 0} \frac{1 - \cos x}{x} = 0$ \qquad \qquad \qquad \qquad
$\lim\limits_{x \to 0} \frac{e^x - 1}{x} = 1$\\
$\lim\limits_{x \to 0} \frac{\log(1 + x)}{x} = 1$ \qquad \qquad \qquad \qquad
$\lim\limits_{x \to 0} \frac{a^x - 1}{x} = \log a$ \qquad \qquad \qquad \qquad
$\lim\limits_{x \to 0} \frac{x}{\tan x} = 1$\\
$\lim\limits_{x \to 0} \frac{\tan x}{x} = 1$ \qquad \qquad \qquad \qquad
$\lim\limits_{x \to 0} \frac{\operatorname{arctanh} x}{x} = 1$ \qquad \qquad \qquad \qquad
$\lim\limits_{x \to 0} \frac{\arctan x}{x} = 1$\\
$\lim\limits_{x \to 0} \frac{\log(1 + x)}{x^2} = \frac{1}{2}$ \qquad \qquad \qquad \qquad
$\lim\limits_{x \to 0} \frac{\log(1 + x)}{x} = 1$ \qquad \qquad \qquad \qquad
$\lim\limits_{x \to 0} \frac{1 - \cos x}{x^2} = \frac{1}{2}$\\
$\lim\limits_{x \to 0} \frac{e^x - 1}{x^2} = 1$ \qquad \qquad \qquad \qquad
$\lim\limits_{x \to 0} \frac{\log(1 + x)}{x^2} = \frac{1}{2}$ \qquad \qquad \qquad \qquad
$\lim\limits_{x \to 0} \frac{a^x - 1}{x^2} = \log a$\\
$\lim\limits_{x \to 0} \frac{\sin x}{x^2} = \frac{1}{6}$ \qquad \qquad \qquad \qquad
$\lim\limits_{x \to 0} \frac{\tan x}{x^2} = \frac{1}{3}$ \qquad \qquad \qquad \qquad
$\lim\limits_{x \to 0} \frac{\arcsin x}{x^2} = \frac{1}{6}$\\
$\lim\limits_{x \to 0} \frac{\arctan x}{x^2} = \frac{1}{3}$ \qquad \qquad \qquad \qquad



\chapter{Serie}

    \section{Serie Numeriche}
    Questo capitolo tratta delle serie numeriche, ovvero di somme infinite di numeri reali.
    
    
    \begin{definizione}
        
    Sia $\{a_n\}$ una successione di numeri reali. La somma parziale di ordine $n$ è definita come:
    \[
    S_n = \sum_{n=1}^n a_n
    \] si dice serie una successione di somme parziali, ovvero:
    \[ S = \sum_{n=1}^{+\infty} a_n \]
    Se esiste il limite della successione delle somme parziali, ovvero: \[ \lim_{n \to +\infty} S_n = S \] allora si dice che la serie converge e si scrive: \[ S = \sum_{k=1}^{+\infty} a_n \] altrimenti si dice che la serie diverge.
    Se la serie converge, si dice che la somma della serie è $S$, 
    Allora la successione $\{a_n\}$ tende a zero,
    Quindi la successione delle somme parziali è limitata.
    Quand'è che una serie converge?
    \end{definizione}
    \subsection{Criteri di Convergenza}
    \begin{itemize}
        \item Se la serie converge, allora la successione $\{a_n\}$ tende a zero.
        \item Se la serie converge, allora la successione delle somme parziali è limitata.
    \end{itemize}
    Per verificare queste condizioni la prima cosa da notare anche ad occhio è se la successione $\{a_n\}$ tende a zero. Se non tende a zero, la serie \emph{non può convergere} perchè sarebbe come sommare infiniti numeri diversi da zero.
    Quando $\{a_n\}$ tende a zero, la serie può comunque non convergere, quindi è condizione necessaria ma non sufficente.

\subsection{La serie geomentrica}
\begin{definizione}
    Una serie geometrica è una serie della forma:
    \begin{equation}
     \sum_{k=1}^{+\infty} x^k = 1 + x + x^2 + \ldots + x^{k} + \ldots 
    \end{equation}
\end{definizione}


    La serie geometrica converge se e solo se $|x| < 1$ e in tal caso la somma della serie è:
    \[ \sum_{n=1}^{+\infty} x^k = \frac{1-x{n+1}}{1-x} \]

    perche facendo il limite si ha:
    \begin{equation}
                \lim_{n \to +\infty} \frac {1-x^{n+1}} {1-x} = \begin{cases}
                    \frac{1}{1-x} & \text{se } -1 < x < 1 \\
                    \text{non esiste} & \text{se } x \leq -1
                \end{cases}
    \end{equation}
    \begin{esempio}[Serie Geometrica]
        Studiamo la serie:
            \[ \sum_{n=1}^{+\infty} \left( \frac{1}{2} \right)^n \]
        Questa è una serie geometrica con $x = 1/2$, quindi converge perchè $|1/2| < 1$ e la somma della serie è:
        \[ \sum_{n=1}^{+\infty} \left( \frac{1}{2} \right)^n = \frac{1}{1-1/2} = 2 \]
    \end{esempio}
    \subsection{Criterio del Confronto}





        \subsection{Criterio del Rapporto}
             \begin{definizione}
                    Sia \emph{$\sum_{n=1}^{+\infty}a_n$} una serie a termini positivi e supponiamo esista il limite:
                    \[ \lim_{n \to +\infty} \frac{a_{n+1}}{a_n} = l \]
                    Allora: se $l < 1$ la serie converge, se $l > 1$ la serie diverge, se $l = 1$ il criterio non ci da alcuna indicazione sul carattere della serie.
            \end{definizione}
                    \emph{Spesso se questo criterio non ci da indicazione sul carattere della serie allora neanche il criterio della radice}.
        \begin{esempio}[Criterio del Rapporto]
            Studiamo la serie:
                \[ \sum_{n=1}^{+\infty} \frac{1}{n!} \]
            Applichiamo il criterio del rapporto:
            in questo caso si ha che:
            \begin{equation} \lim_{n \to +\infty} \frac{1/(n+1)!}{1/n!} = \lim_{n \to +\infty} \frac{1}{n+1} = 0
            \end{equation}
            Quindi per il criterio del rapporto $0<1$ quindi la serie converge.
        \end{esempio}
        \subsection{Criterio della Radice}
        \begin{definizione}
            Sia \emph{$\sum_{n=1}^{+\infty}a_n$} una serie a termini positivi e supponiamo esista il limite:
            \[ \lim_{n \to +\infty} \sqrt[n]{a_n} = l \]
            Allora: se $l < 1$ la serie converge, se $l > 1$ la serie diverge, se $l = 1$ il criterio non ci da alcuna indicazione sul carattere della serie.
        \end{definizione} 
        
    \emph{Spesso se questo criterio non ci da indicazione sul carattere della serie allora neanche il criterio del rapporto}


    \begin{esempio}[Criterio della Radice]
        Studiamo la serie:
            \[ \sum_{n=1}^{+\infty} \frac{1}{n^n} \]
        Applichiamo il criterio della radice:
        in questo caso si ha che:
        \begin{equation} \lim_{n \to +\infty} \sqrt[n]{1/n^n} = \lim_{n \to +\infty} \frac{1}{n} = 0
        \end{equation}
        Quindi per il criterio della radice $0<1$ quindi la serie converge.
            \end{esempio}
    \subsection{Criterio di Cauchy}
            \begin{definizione}
                    Sia $\{a_n\}$ una successione di numeri reali. La serie $\sum_{k=1}^{+\infty} a_k$ converge se e solo se per ogni $\varepsilon > 0$ esiste $n_0 \in \mathbb{N}$ tale che per ogni $n \geq m \geq n_0$ si ha:
                      \[ |a_{m+1} + a_{m+2} + \ldots + a_n| < \varepsilon \]
            \end{definizione}
    \begin{esempio}[Criterio di Cauchy]
        Studiamo la serie:
            \[ \sum_{n=1}^{+\infty} \frac{1}{n^2} \]
        Applichiamo il criterio di Cauchy:
        in questo caso si ha che:
        \begin{equation} |a_{m+1} + a_{m+2} + \ldots + a_n| = \left| \frac{1}{(m+1)^2} + \frac{1}{(m+2)^2} + \ldots + \frac{1}{n^2} \right| < \varepsilon
        \end{equation}
        Quindi per il criterio di Cauchy la serie converge.
    \end{esempio}

\end{document}
    